\documentclass[journal,12pt,twocolumn]{IEEEtran}
\usepackage{graphicx}
\usepackage{paralist}
\usepackage{setspace}
\usepackage{gensymb}
\singlespacing
\usepackage[cmex10]{amsmath}
\usepackage{amsthm}
\usepackage{amsmath}
\usepackage{amssymb}
\usepackage{mathrsfs}
\usepackage{txfonts}
\usepackage{stfloats}
\usepackage{bm}
\usepackage{cite}
\usepackage{cases}
\usepackage{subfig}
\usepackage{longtable}
\usepackage{multirow}

\usepackage{enumitem}
\usepackage{mathtools}
\usepackage{steinmetz}
\usepackage{tikz}
\usepackage{circuitikz}
\usepackage{verbatim}
\usepackage{tfrupee}
\usepackage[breaklinks=true]{hyperref}
\usepackage{graphicx}
\usepackage{tkz-euclide}

\usetikzlibrary{calc,math}
\usepackage{listings}
    \usepackage{color}                                            %%
    \usepackage{array}                                            %%
    \usepackage{longtable}                                        %%
    \usepackage{calc}                                             %%
    \usepackage{multirow}                                         %%
    \usepackage{hhline}                                           %%
    \usepackage{ifthen}                                           %%
    \usepackage{lscape}     
\usepackage{multicol}
\usepackage{chngcntr}

\DeclareMathOperator*{\Res}{Res}

\renewcommand\thesection{\arabic{section}}
\renewcommand\thesubsection{\thesection.\arabic{subsection}}
\renewcommand\thesubsubsection{\thesubsection.\arabic{subsubsection}}

\renewcommand\thesectiondis{\arabic{section}}
\renewcommand\thesubsectiondis{\thesectiondis.\arabic{subsection}}
\renewcommand\thesubsubsectiondis{\thesubsectiondis.\arabic{subsubsection}}


\hyphenation{op-tical net-works semi-conduc-tor}
\def\inputGnumericTable{}                                 %%

\lstset{
%language=C,
frame=single, 
breaklines=true,
columns=fullflexible
}
\begin{document}


\newtheorem{theorem}{Theorem}[section]
\newtheorem{problem}{Problem}
\newtheorem{proposition}{Proposition}[section]
\newtheorem{lemma}{Lemma}[section]
\newtheorem{corollary}[theorem]{Corollary}
\newtheorem{example}{Example}[section]
\newtheorem{definition}[problem]{Definition}

\newcommand{\BEQA}{\begin{eqnarray}}
\newcommand{\EEQA}{\end{eqnarray}}
\newcommand{\define}{\stackrel{\triangle}{=}}
\bibliographystyle{IEEEtran}
\raggedbottom
\setlength{\parindent}{0pt}
\providecommand{\mbf}{\mathbf}
\providecommand{\pr}[1]{\ensuremath{\Pr\left(#1\right)}}
\providecommand{\qfunc}[1]{\ensuremath{Q\left(#1\right)}}
\providecommand{\sbrak}[1]{\ensuremath{{}\left[#1\right]}}
\providecommand{\lsbrak}[1]{\ensuremath{{}\left[#1\right.}}
\providecommand{\rsbrak}[1]{\ensuremath{{}\left.#1\right]}}
\providecommand{\brak}[1]{\ensuremath{\left(#1\right)}}
\providecommand{\lbrak}[1]{\ensuremath{\left(#1\right.}}
\providecommand{\rbrak}[1]{\ensuremath{\left.#1\right)}}
\providecommand{\cbrak}[1]{\ensuremath{\left\{#1\right\}}}
\providecommand{\lcbrak}[1]{\ensuremath{\left\{#1\right.}}
\providecommand{\rcbrak}[1]{\ensuremath{\left.#1\right\}}}
\theoremstyle{remark}
\newtheorem{rem}{Remark}
\newcommand{\sgn}{\mathop{\mathrm{sgn}}}
\providecommand{\abs}[1]{\left\vert#1\right\vert}
\providecommand{\res}[1]{\Res\displaylimits_{#1}} 
\providecommand{\norm}[1]{\left\lVert#1\right\rVert}
%\providecommand{\norm}[1]{\lVert#1\rVert}
\providecommand{\mtx}[1]{\mathbf{#1}}
\providecommand{\mean}[1]{E\left[ #1 \right]}
\providecommand{\fourier}{\overset{\mathcal{F}}{ \rightleftharpoons}}
%\providecommand{\hilbert}{\overset{\mathcal{H}}{ \rightleftharpoons}}
\providecommand{\system}{\overset{\mathcal{H}}{ \longleftrightarrow}}
	%\newcommand{\solution}[2]{\textbf{Solution:}{#1}}
\newcommand{\comb}[2]{{}^{#1}\mathrm{C}_{#2}}
\newcommand{\solution}{\noindent \textbf{Solution: }}
\newcommand{\cosec}{\,\text{cosec}\,}
\newcommand{\cosec}{}
\providecommand{\dec}[2]{\ensuremath{\overset{#1}{\underset{#2}{\gtrless}}}}
\newcommand{\myvec}[1]{\ensuremath{\begin{pmatrix}#1\end{pmatrix}}}
\newcommand{\mydet}[1]{\ensuremath{\begin{vmatrix}#1\end{vmatrix}}}
\numberwithin{equation}{subsection}
\makeatletter
\@addtoreset{figure}{problem}
\makeatother
\let\StandardTheFigure\thefigure
\let\vec\mathbf
\renewcommand{\thefigure}{\theproblem}
\def\putbox#1#2#3{\makebox[0in][l]{\makebox[#1][l]{}\raisebox{\baselineskip}[0in][0in]{\raisebox{#2}[0in][0in]{#3}}}}
     \def\rightbox#1{\makebox[0in][r]{#1}}  
     \def\centbox#1{\makebox[0in]{#1}}
     \def\topbox#1{\raisebox{-\baselineskip}[0in][0in]{#1}}
     \def\midbox#1{\raisebox{-0.5\baselineskip}[0in][0in]{#1}}
\vspace{3cm}
\title{Assignment 6}
\author{Gaureesha Kajampady - EP20BTECH11005}
\maketitle  
\newpage
\bigskips
\renewcommand{\thefigure}{\theenumi}
\renewcommand{\thetable}{\theenumi}
Download latex-tikz codes from 
%
\begin{lstlisting}
https://github.com/gaureeshk/assignment6_2/blob/main/assignment6.tex
\end{lstlisting}
\section{Problem}
gov/stats/2018/STATISTICS-PAPER-2 , Q.58\\
Let $X_{1}$ and  $X_{2}$ be i.i.d random variables with poisson. Then ($X_{1}+2X_{2}$) is not sufficient because
\begin{enumerate}
    \item{$\pr{X_{1}=1,X_{2}=1 |T=3}$ depends on $\lambda$}\\
    \item{$X_{1}+2X_{2}$ is poisson}\\
     \item{$X_{1}+2X_{2}$ is not poisson}\\
      \item{$\pr{X_{1}=1,X_{2}=1 | T=3}$ is Poisson with parameter one}\\
\end{enumerate}
Where T=($X_{1}+2X_{2}$)
\section{Solution} 
\begin{definition}
     Statistic : A statistic is a function T = r($X_{1},X_{2},\dots,X_{n}$) of the random sample $X_{1},X_{2},\dots,X_{n}$.
   \end{definition}
   \begin{definition}
    Sufficient Statistics : A statistic t = T(X) is sufficient for $\theta$ if the conditional probability distribution of data X, given the statistic t = T(X), doesn't depend on the parameter $\theta$.
   \end{definition}
   \begin{theorem}[Factorization theorem]\label{1}
    : Let $X_{1},X_{2} · · · , X_{n}$ form a random sample from either a continuous
distribution or a discrete distribution for which the pdf or the point mass function is $f(x\vert\theta)$,
where the value of $\theta$ is unknown and belongs to a given parameter space $\Theta$. A statistic
T($X_{1},X_{2} · · · , X_{n}$) is a sufficient statistic for $\theta$ if and only if the joint pdf or the joint point mass
function $f_{n}$(x$\vert\theta$) $X_{1},X_{2} · · · , X_{n}$ can be factorized as follows for all values of x = ($X_{1},X_{2} · · · , X_{n}$) \to
$R^{n}$ and all values of $\theta \in \Theta$:
$f_{n}(x\vert \theta) = u(x)v[T(x), \theta].$\\
Here the function u may depend on x but does not
depend on $\theta$, and the function v depends on $\theta$ but will depend on the observed value x only through the value of the statistic T(x).
   \end{theorem}
   \begin{enumerate}
  \item We know T=3 when $(X_{1},X_{2})$ have values (1,1) and (3,0)
   \begin{multline}
&\pr{X_{1}=1,X_{2}=1 |T=3}\\&=\frac{\pr{X_{1}=1,X_{2}=1 \cap T=3}}{\pr{T=3}}\\
&=\frac{\pr{X_{1}=1,X_{2}=1}}{\pr{X_{1}=1,X_{2}=1}+\pr{X_{1}=3,X_{2}=0}}\\
&=\frac{e^{-2\lambda}\lambda^{2}}{e^{-2\lambda}\lambda^{2}+\frac{e^{-3\lambda}\lambda^{3}}{6}}\\
&=\frac{6}{6+\lambda}\neq \frac{e^{-1}1^{\lambda}}{\lambda!}
\end{multline}
Hence $\pr{X_{1}=1,X_{2}=1 |T=3}$ depends on $\lambda$ but is not poisson with parameter 1.
$\implies$option 4 is incorrect and option 1 is correct.\\
\item Now to find characteristic function of T,\\
Let $\Phi_{X_{1}}(\omega)$, $\Phi_{2X_{2}}(\omega)$ and  $\Phi_{T}(\omega)$be the characteristic functions of probability density function of random variables $X_{1}$, $2X_{2}$ and T respectively.\\
 \begin{multline}
     $\Phi_{X_{1}}(\omega)$=E(e^{i\omega X_{1}})&=\sum_{x=0}^{\infty}\pr{X_{1}=x}e^{i\omega x}\\
                     &=\sum_{x=0}^{\infty}\frac{e^{i\omega x-\lambda}\lambda^{x}}{x!}\\
                     &=e^{-\lambda}\sum_{x=0}^{\infty}\frac{(e^{i\omega}\lambda)^{x}}{x!}\\
                     &=e^{\lambda(e^{i\omega}-1)}
 \end{multline}
 Similarly,
 \begin{multline}
     $\Phi_{2X_{2}}(\omega)$=E(e^{i\omega 2X_{2}})&=\sum_{x=0}^{\infty}\pr{X_{2}=\frac{x}{2}}e^{i\omega x}\\
                     &=\sum_{x=0,2,4....}^{\infty}\frac{e^{i\omega x-\lambda}\lambda^{\frac{x}{2}}}{(\frac{x}{2})!}\\
                     &=e^{-\lambda}\sum_{\frac{x}{2}=0,1,2..}^{\infty}\frac{(e^{2i\omega}\lambda)^{\frac{x}{2}}}{(\frac{x}{2})!}\\
                     &=e^{\lambda(e^{2i\omega}-1)}\\
                     \end{multline}
                     \begin{align}
 \Phi_{T}(\omega)&=\Phi_{X_{1}}(\omega)\times\Phi_{2X_{2}}(\omega)\\
  &=e^{\lambda(e^{i\omega}+e^{2i\omega}-2)}\\ 
  &\neq e^{\mu(e^{i\omega}-1)}
  \end{align}
  Hence the characteristic function of T ($\Phi_{T}(\omega)$) is not in the form of the characteristic function of a poisson random variable (for any value of the parameter $\mu$).\\
Hence $T=X_{1}+2X_{2}$ is not poisson distribution.\\
$\implies$ option 2 is incorrect, option 3 is correct\\
Hence options 1 and 3 are correct.
\end{document}





















